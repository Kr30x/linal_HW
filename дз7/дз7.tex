\documentclass[12pt, a4paper]{article}
\usepackage{gset}
\setcounter{MaxMatrixCols}{50}
\begin{document}
	\maintitle{ЛАиГ. Домашнее задание 7}
	\textbf{1. П213 Как изменится определитель порядка п, если его строки: написать в обратном порядке?} \sspace
	Пусть какое-то произведение входило в сумму определителя и ему соответствовала какая-то перестановка. В новый определитель то же самое произведение будет соответствовать той же перестановке, которую записали в обратном порядке. То есть, чтобы вычислить знак произведения, нужно найти знак новой перестнановки. Ее можно получить из изначальной путем транспозиций 1 и последнего, второго и предпоследнего и так далее. Всего таких транспозиций $\lfloor \frac{n}{2} \rfloor$. У каждой транспозиции знак -1, поэтому знак новой перестановки будет сохраняться, если $\lfloor \frac{n}{2} \rfloor \equiv 0 \mod 2$ и меняться, если $\lfloor \frac{n}{2} \rfloor \equiv 1 \mod 2$. Так как либо у всех произведений знак меняется, либо у всех он сохраняется, то и для всего определителя верно, что $\det A = \det A'$, если $\lfloor \frac{n}{2} \rfloor \equiv 0 \mod 2$ и $\det A = -\det A'$, если $\lfloor \frac{n}{2} \rfloor \equiv 1 \mod 2$ \sspace
	\textbf{2. Как изменится определитель матрицы, если её «транспонировать» (то есть отразить) относительно побочной диагонали?} \sspace
	Транспонирование относительно побочной диагонали можно реализовать с помощью преобразования из прошлой задачи, назовем его функцией С. Тогда транспонирование относительно побочной диагонали для матрицы $A = C(C(A)^T)$ Так как транспонирование не влияет на знак определителя, как не влияет и четное количество применений функций С (размер матрицы не меняется, значит либо определитель не меняется 2n раз, либо знак у него меняется 2n раз), то определитель матрицы не меняется при транспонировании относительно побочной диагонали. \sspace
	\textbf{3. Как изменится определитель матрицы A, если при всех i, j элемент $a_{ij}$ умножить на $c^{i-j}$, где $c \neq 0$ — некоторое фиксированное число?} \sspace
	Обозначим диагонали степенями c, на которое надо домножить ее элементы. Главная диагональ будет 0, диаггональ над ней 1, под главной -1 и так далее. Хотим доказать, что для каждой перестановки верно, что если сложить номера диагоналей элементов, то получится 0. Тогда итоговое произведение перестановки не изменится. Сумму номеров диагоналей можно вычислить как $\sum_{i=1}^{n} (a_i - i)$. Например, если матрица 3 на 3, то для перестановки $1,2,3$ сумма номеров диагоналей будет $0 + 0 + 0$, а для перестановки $2,3,1$ будет $1 + 1 - 2$. Так как $\sum_{i=1}^{n} (a_i - i )= \sum_{i = 1}^{n} a_i - \sum_{i = 1}^{n} i$. $\sum_{i = 1}^{n} a_i = \sum_{i = 1}^{n} i = \frac{(n + 1)n}{2} \Rightarrow \sum_{i=1}^{n} (a_i - i) = 0$. То есть если сложить все степени с, на которое надо умножить произведение, всегда получится 0. То есть никакое произведение не изменится. Значит и весь определитель не изменится. \sspace
	\textbf{4. П227, П228} \sspace
	\textbf{П227 Доказать, что определитель не изменится, если из каждой строки, кроме последней, вычесть все последующие строки.} \sspace
	Уже известно, что определитель не меняется при элементарных преобразованиях 1 типа. Так как преобразования из условия можно выполнить с помощью последовательного применения элементарных преобразований 1 типа, то определитель матрицы не меняется при преобразовании из условия. \sspace
	\textbf{П228 Доказать, что определитель не изменится, если к каждому столбцу, начиная со второго, прибавить все предыдущие столбцы.} \sspace
	Уже известно, что определитель матрицы не меняется при транспонировании и элементарных преобразоаваниях первого типа. Тогда так как преобразование из условия можно представить как транспонирование $\rightarrow$ последовательное применение элементарных преобразований 1 типа $\rightarrow$ транспонирование, то определитель не изменится. \sspace
	\textbf{5. П229 Как изменится определитель, если из каждой строки, кроме последней, вычесть последующую строку, из последней строки вычесть прежнюю первую строку?} \sspace
	Определитель никак не изменится, потому что будут использованы только преобразования 1 типа. \sspace
	\textbf{6. П236, П240} \sspace
	\textbf{П236 Разлагая по 3-ей строке, вычислить определитель} \sspace
	$
	\detfour{2  & -3 & 4 & 1}{4 & -2 & 3 & 2}{a & b & c & d}{3 & -1 & 4 & 3} = a * \detthree{-3 & 4 & 1}{-2 & 3 & 2}{-1 & 4 & 3} - b * \detthree{2 & 4 & 1}{4 & 3 & 2}{3 & 4 & 3} + c * \detthree{2 & -3 & 1}{4 & -2 & 2}{3 & -1 & 3} - d * \detthree{2 & -3 & 4}{4 & -2 & 3}{3 & -1 & 4} = \sspace = 8a + 15b + 12c - 19d \sspace
	$
	\textbf{П240 Вычислить определитель} \sspace
	$
	\detfive{x & a & b & 0 & c}{0 & y & 0 & 0 & d}{0 & e & z & 0 & f}{g & h & k & u & l}{0 & 0 & 0 & 0 & v} \sspace
	$
	Разложим по 5 строке, затем по 2 строке, затем снова по 2 строке \sspace
	$
	\detfive{x & a & b & 0 & c}{0 & y & 0 & 0 & d}{0 & e & z & 0 & f}{g & h & k & u & l}{0 & 0 & 0 & 0 & v} = v * \detfour{x & a & b & 0}{0 & y & 0 & 0}{0 & e & z & 0}{g & h &k & u} = v * y * \detthree{x & b & 0}{0 & z & 0}{g & k & u} = v * y * z * \dettwo{x & 0}{g & u} =  v * y * z * x * u \bs
	$
	\textbf{7. П260, П263} \sspace
	\textbf{П260 Вычислить определитель} \sspace
	$
	\detfour{\rowfor{-3}{9}{3}{6}}{\rowfor{-5}{8}{2}{7}}{\rowfor{4}{-5}{-3}{-2}}{\rowfor{7}{-8}{-4}{-5}} \rsa 
	\detfour{\rowfor{1}{4}{0}{4}}{\rowfor{-5}{8}{2}{7}}{\rowfor{4}{-5}{-3}{-2}}{\rowfor{7}{-8}{-4}{-5}} \rsa
	\detfour{\rowfor{1}{4}{0}{4}}{\rowfor{0}{28}{2}{27}}{\rowfor{0}{-21}{-3}{-18}}{\rowfor{0}{-36}{-4}{-33}} \rsa
	\detfour{\rowfor{1}{4}{0}{4}}{\rowfor{0}{7}{-1}{9}}{\rowfor{0}{-21}{-3}{-18}}{\rowfor{0}{-36}{-4}{-33}} \rsa \bs \rsa
	\detfour{\rowfor{1}{4}{0}{4}}{\rowfor{0}{7}{-1}{9}}{\rowfor{0}{0}{-6}{9}}{\rowfor{0}{-1}{-9}{12}} \rsa
	-\detfour{\rowfor{1}{4}{0}{4}}{\rowfor{0}{7}{-1}{9}}{\rowfor{0}{0}{-6}{9}}{\rowfor{0}{1}{9}{-12}} \rsa
	\detfour{\rowfor{1}{4}{0}{4}}{\rowfor{0}{1}{9}{-12}}{\rowfor{0}{0}{-6}{9}}{\rowfor{0}{7}{-1}{9}} \rsa
	\detfour{\rowfor{1}{4}{0}{4}}{\rowfor{0}{1}{9}{-12}}{\rowfor{0}{0}{-6}{9}}{\rowfor{0}{0}{-64}{93}} \rsa \bs \rsa
	\detfour{\rowfor{1}{4}{0}{4}}{\rowfor{0}{1}{9}{-12}}{\rowfor{0}{0}{0}{-9}}{\rowfor{0}{0}{2}{-6}} \rsa
	-\detfour{\rowfor{1}{4}{0}{4}}{\rowfor{0}{1}{9}{-12}}{\rowfor{0}{0}{2}{-6}}{\rowfor{0}{0}{0}{-9}} = 18 \bs
	$
	\textbf{П263} \sspace
	$
	\detfour{\rowfor{3}{-3}{-2}{-5}}{\rowfor{2}{5}{4}{6}}{\rowfor{5}{5}{8}{7}}{\rowfor{4}{4}{5}{6}} \rsa
	\detfour{\rowfor{1}{-8}{-6}{-11}}{\rowfor{2}{5}{4}{6}}{\rowfor{5}{5}{8}{7}}{\rowfor{4}{4}{5}{6}} \rsa
	\detfour{\rowfor{1}{-8}{-6}{-11}}{\rowfor{0}{21}{16}{28}}{\rowfor{0}{45}{38}{62}}{\rowfor{0}{36}{29}{50}} \rsa
	\detfour{\rowfor{1}{-8}{-6}{-11}}{\rowfor{0}{21}{16}{28}}{\rowfor{0}{3}{6}{6}}{\rowfor{0}{36}{29}{50}} \rsa \bs \rsa
	\detfour{\rowfor{1}{-8}{-6}{-11}}{\rowfor{0}{0}{-26}{-14}}{\rowfor{0}{3}{6}{6}}{\rowfor{0}{0}{-43}{-22}} \rsa
	-\detfour{\rowfor{1}{-8}{-6}{-11}}{\rowfor{0}{3}{6}{6}}{\rowfor{0}{0}{-26}{-14}}{\rowfor{0}{0}{-43}{-22}} \rsa
	-\detfour{\rowfor{1}{-8}{-6}{-11}}{\rowfor{0}{3}{6}{6}}{\rowfor{0}{0}{17}{8}}{\rowfor{0}{0}{8}{2}}\rsa \bs \rsa
	-\detfour{\rowfor{1}{-8}{-6}{-11}}{\rowfor{0}{3}{6}{6}}{\rowfor{0}{0}{1}{4}}{\rowfor{0}{0}{0}{-30}} = 1 * 3 * 1 * 30 = 90
	$\bs
	\textbf{8. Даны матрицы $A, B \in M_4(\mathbb{R})$. Известно, что det A = 1 и \\
		\[B^{(1)} = 3A^{(3)} - 2A^{(4)} \quad B^{(2)} = 2A^{(1)} + 3A^{(2)}\quad B^{(3)} = -2A^{(1)} + 2A^{(3)} + A^{(4)} \quad B^{(4)} = 2A^{(2)} + 3A^{(4)}\]
		\\
		где $A^{(i)} и B^{(j)}$ обозначают i-й столбец матрицы A и j-й столбец матрицы B соответственно. Найдите det B.} \sspace
		Получим матрицу B из матрицы А с помощью элементарных преобразований. При преобразовании 1 типа ничего не изменится, при преобразовании 3 типа умножим определитель на обратно константе число \sspace
		$\det A = \det (A^{(1)}, A^{(2)}, A^{(3)}, A^{(4)}) = \frac{1}{3} \det (A^{(1)}, A^{(2)}, 3A^{(3)} - 2A^{(4)}, A^{(4)}) = \sspace= \frac{1}{9} \det (A^{(1)}, 3A^{(2)} + 2A^{(1)}, 3A^{(3)} - 2A^{(4)}, A^{(4)}) = \sspace = \frac{-1}{18} \det (-2A^{(1)} + 2A^{(3)}, 3A^{(2)} + 2A^{(1)}, 3A^{(3)} - 2A^{(4)}, A^{(4)}) = \sspace = 
		\frac{-1}{54} \det (-2A^{(1)} + 2A^{(3)} + A^{(4)}, 3A^{(2)} + 2A^{(1)}, 3A^{(3)} - 2A^{(4)}, 3A^{(4)} + 2A^{(2)}) = \sspace = \frac{1}{54} \det (3A^{(3)} - 2A^{(4)}, 3A^{(2)} + 2A^{(1)}, -2A^{(1)} + 2A^{(3)} + A^{(4)}, 3A^{(4)} + 2A^{(2)}) = \sspace =
		\frac{1}{54} \det B = 1 \Rightarrow \det B = 54$ 
	 \end{document}