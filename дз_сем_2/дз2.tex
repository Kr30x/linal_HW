\documentclass[12pt, a4paper]{article}
\usepackage{gset}
\begin{document}
	\maintitle{ЛАиГ. Домашнее задание 2}
	1) П809 \\
	\\
	$\matthree{4}{3}{-3}{2}{3}{-2}{4}{4}{-3} = \matthree{1}{3}{1}{2}{2}{1}{3}{4}{2} * \matthree{1}{0}{0}{0}{2}{0}{0}{0}{1} * \matthree{0}{2}{-1}{1}{1}{-1}{-2}{-5}{4}$\bspace
	$\matthree{1}{3}{1}{2}{2}{1}{3}{4}{2} * \matthree{0}{2}{-1}{1}{1}{-1}{-2}{-5}{4} = \matthree{0 + 3 - 2}{2 + 3 - 5}{-1 -3 + 4}{0 + 2 - 2}{4 + 2 - 5}{-2 -2  + 4}{0 + 4 - 4}{6 + 4 -10}{- 3 - 4 + 8} = \matthree{1}{0}{0}{0}{1}{0}{0}{0}{1} = \bspace  = E$
	\bspace 
	$\matthree{1}{3}{1}{2}{2}{1}{3}{4}{2}  = A; \matthree{0}{2}{-1}{1}{1}{-1}{-2}{-5}{4} = A^{-1}; \matthree{1}{0}{0}{0}{2}{0}{0}{0}{1} = B$\bspace
	$\matthree{4}{3}{-3}{2}{3}{-2}{4}{4}{-3}  = A* B * A^{-1} $\bspace
	$(A* B * A^{-1}) ^ 6 = A * B ^ 6 * A ^ {-1 }$\bspace
	$B^6 = \matthree{1}{0}{0}{0}{2}{0}{0}{0}{1} ^ 6 = \matthree{1}{0}{0}{0}{64}{0}{0}{0}{1}$ \bspace
	$\matthree{4}{3}{-3}{2}{3}{-2}{4}{4}{-3} ^ 6 = \matthree{1}{3}{1}{2}{2}{1}{3}{4}{2} * \matthree{1}{0}{0}{0}{64}{0}{0}{0}{1} * \matthree{0}{2}{-1}{1}{1}{-1}{-2}{-5}{4} = \matthree{1}{192}{1}{2}{128}{1}{3}{256}{2} * \bspace \matthree{0}{2}{-1}{1}{1}{-1}{-2}{-5}{4} = \matthree{190}{189}{-189}{126}{127}{-126}{252}{252}{-251}$
	\bspace
	2) П815 \\
	а) $(A + B) ^ 2 = A ^ 2 + AB + BA + B ^ 2$\\
	Так как $AB \neq BA$, то $AB + BA \neq 2AB$, тогда $A ^ 2 + AB + BA + B ^ 2 \neq A ^ 2 + 2AB  + B ^ 2$\\
	б) $(A + B)(A - B) = A ^ 2 + BA - AB - B ^ 2$\\
	Так как $AB \neq BA$, то $BA - AB \neq 0$, тогда $A ^ 2 + BA - AB - B ^ 2 \neq A ^ 2 + B ^ 2$\bspace
	3) П832\\
	Пусть $A = \mattwo{a}{b}{c}{d}$, $A ^ 2 = \mattwo{a ^ 2  + b * c}{a * b + b * d}{a * c + c * d}{b * c + d ^ 2}$\bspace
	Если $A ^ 2 = \zeromat$, то $\left \{ \begin{array}{c} a ^ 2 + b  * c = 0 \\ a * b + b * d = 0 \\ a * c + c * d  = 0 \\ b * c + d ^ 2 = 0 \end{array} \right . $ 
	$\left \{ \begin{array}{c} a ^ 2 + b  * c = 0 \\ b (a + d) = 0 \\ c (a + d)  = 0 \\ b * c + d ^ 2 = 0 \end{array} \right . $ \bspace
	Пусть $a = 0$, тогда, чтобы $ A ^ 2 = \zeromat$, необходимо, чтобы $b * c = 0$ (из 1), тогда и $d ^ 2 = 0$ (из 4). Следовательно $A = \mattwo{0}{b}{0}{0}$ или $A = \mattwo{0}{0}{c}{0}$ \bspace
	Пусть $a \neq 0$, тогда $b * c = -a ^ 2$ (из 1), тогда $b \neq 0 \land c \neq 0$, тогда $d = -a$ из \\ (2 или 3). Тогда $A = \mattwo{a}{b}{-\frac{a}{b}}{-a}$\bspace
	
	4) Квадратная матрица A называется верхнетреугольной, если aij = 0 при всех i > j, то есть все её элементы ниже главной диагонали равны нулю. Докажите, что произведение двух верхне- треугольных матриц снова будет верхнетреугольной матрицей. \bspace
	
	$A_{m, n} = \mtrfor{\rowfor{a_{1,1}}{a_{1,2}}{\cdots}{a_{1,n}}}{\rowfor{0}{a_{2,2}}{\cdots}{\vdots}}{\rowfor{0}{0}{\ddots}{a_{m - 1,n}}}{\rowfor{0}{0}{0}{a_{m,n}}}$ $B_{n, p} = \mtrfor{\rowfor{b_{1,1}}{b_{1,2}}{\cdots}{b_{1,p}}}{\rowfor{0}{b_{2,2}}{\cdots}{\vdots}}{\rowfor{0}{0}{\ddots}{b_{n - 1,n}}}{\rowfor{0}{0}{0}{b_{n,p}}}$ \bspace
	Пусть $C = A * B$. Рассмотрим $c_{i, j} = A ^ {(i)} * B_{(j)} = a_{i, 1} * b_{1, j} + a_{i, 2} * b_{2, j} + \cdots + a_{i, n} * b_{n, j}$. Все элементы $a_{i, k} = 0$, при $0 < k < i$. Тогда первый ненулевой элемент $a_{i, k} = a_{i,i}$. То есть первые $i - 1$ произведений "занулятся". Аналогично для B, последний ненулевой элемент в $B_{(j)} = b_{j,j}$. То есть последние $n - j$ прозведений "занулятся". Чтобы $c_{i,j} \neq 0$, необходимо, чтобы был хотя бы 1 ненулевое слагаемое. Иными словами $i  - 1 + n - j < n = i < j + 1$. Элементы с такими индексами находятся на главной диагонали или выше, значит $C$ - верхнетреугольная матрица. \bspace
	
	5) Назовём квадратную матрицу побочно-диагональной, если все её элементы вне побочной диаго- нали равны нулю. Пусть A, B — две побочнодиагональные матрицы, причём у матрицы A (соответственно B) на побочной диагонали стоят элементы $\lambda_1, \lambda_2, . . . , \lambda_n$ (соответственно$ \mu_1, \mu_2, . . . , \mu_n)$, если считать от правого верхнего угла к левому нижнему. Найдите произведения AB и BA и определите условия, при которых матрицы A и B коммутируют (то есть AB = BA). \bspace
	
	$A = \mtrfor{\rowfor{0}{0}{0}{\lambda_1}}{\rowfor{0}{0}{\lambda_2}{0}}{\rowfor{0}{\iddots}{0}{0}}{\rowfor{\lambda_n}{0}{0}{0}} 
	B = \mtrfor{\rowfor{0}{0}{0}{\mu_1}}{\rowfor{0}{0}{\mu_2}{0}}{\rowfor{0}{\iddots}{0}{0}}{\rowfor{\mu_n}{0}{0}{0}} 
	$\bspace
	
	$
	AB = \mtrfor{\rowfor{\lambda_1 * \mu_n}{0}{0}{0}}{\rowfor{0}{\lambda_2 * \mu_{n - 1}}{0}{0}}{\rowfor{0}{0}{\ddots}{0}}{\rowfor{0}{0}{0}{\lambda_n * \mu_1}}    
	BA =  \mtrfor{\rowfor{\mu_1 * \lambda_n}{0}{0}{0}}{\rowfor{0}{\mu_2 * \lambda_{n - 1}}{0}{0}}{\rowfor{0}{0}{\ddots}{0}}{\rowfor{0}{0}{0}{\mu_n * \lambda_1}} 
	$\bspace
	Когда $AB = BA$? Это условие выполняется, если \\
	 \[\lambda_1 * \mu_n = \lambda_n * \mu_1, \lambda_2 * \mu_{n - 1} = \lambda_{n - 1} * \mu_2, \cdots, \lambda_{n} * \mu_1 = \lambda_1 * \mu_n \]\\
	Получается, что $AB = BA$, когда $\lambda_i * \mu_{n - i + 1} = \lambda_{n - i + 1} * \mu_i$, то есть $\frac{\lambda_i}{\lambda_{n - i + 1}} = \frac{\mu_i}{\mu_{n - i + 1}}$ для любого $1 \leq i \leq n$. \bspace
	
	6) Найдите все квадратные (n × n)-матрицы X, удовлетворяющие соотношению XA = AX для любой квадратной матрицы A того же размера. \bspace
	
	$\matA \matX$ \bspace
	Если Х коммутирует со всеми матрицами, то Х коммутирует с диагональными  матрицами. Тогда Х - диагональная матрица. Докажем, что $X = tE$. Пусть $X_{i, i} \neq X_{j, j}$. Тогда $AX_{i, j} = A_{(i)} * X ^ {(j)} = a_{i, 1} * x_{1, j} + a_{i, 2} * x_{2, j} + \cdots + a_{i, n} * x_{n, j} = a_{i, j} * x_{i, i}$\\
	 $XA_{i,j} = X_{(i)} * A^{(j)} = x_{i, i} * a_{i, j}$. \\
	 Так как Х коммутирует с любой матрицей, возьмем такую матрицу А, чтобы $a_{i, j} \neq 0$. Тогда 	$AX_{i, j}  \neq XA_{i,j}$. Значит Х не коммутирует с А. Противоречие. Значит $X = tE$. \bspace
	
	7) Приведите пример трёх квадратных (2 × 2)-матриц A, B, C, для которых $tr(ABC) \neq tr(ACB)$. \bspace
	
	$A = \mattwo{3}{2}{3}{1}$
	$B = \mattwo{1}{3}{2}{1}$
	$C = \mattwo{1}{1}{2}{2}$ \bspace
	$ABC = \mattwo{29}{29}{25}{25}$
	$ACB = \mattwo{21}{28}{15}{20}$\bspace
	$tr(ABC) = 54, tr(ACB) = 41$
	
\end{document}