\documentclass[12pt, a4paper]{article}
\usepackage{gset}
\setcounter{MaxMatrixCols}{50}
\begin{document}
	\maintitle{ЛАиГ. Домашнее задание 6}
	\textbf{1)} \sspace
	$(12)(345)(67) = ((1627)(354))^2$ \\
	$(1234)(5678) = (15263748)^2$\\
	$(1234567) = (1526374)^2$ \sspace
	\textbf{2)} П159, П160 \sspace
	$\sigma = \begin{pmatrix} 1 & 2 & 3 & 4  & \cdots & 2n - 3 & 2n-2 & 2n-1 & 2n \\ 3 & 4 & 5 & 6  & \cdots & 2n-1 & 2n & 1 & 2\end{pmatrix} = (1, 3, 5, \cdots, 2n - 1) * (2, 4, 6, \cdots, 2n)$ \sspace
	$dec(\sigma) = 2n - 2 \Rightarrow sgn(\sigma) = 1$ \sspace
	$\sigma = \begin{pmatrix} 1 & 2 & 3 & 4 & 5 & 6 & \cdots & 3n - 2 & 3n-1 & 3n \\ 2 & 3 & 1 & 5 & 6 & 4  &\cdots& 3n-1 & 3n & 3n-2\end{pmatrix} = \sspace = (1,2,3)*(4*5*6)*\cdots*(3n-2, 3n-1, 3n)$ \sspace
	$dec(\sigma) = n$, если $n$ - четное, то $sgn(\sigma) = 1$, иначе $sgn(\sigma) = -1$ \sspace
	\textbf{3) П1, П5, П6, П9} \sspace
	П1) $\left | \begin{array}{rr} 5 & 2\\ 7 & 3 \end{array} \right | = 5*3 - 7*2 = 1$ \sspace
	П5) $\left | \begin{array}{rr} a^2 & ab\\ ab & b^2 \end{array} \right | = a^2b^2 - (ab)^2 = 0$ \sspace
	П6) $\left | \begin{array}{rr} n + 1 & n\\ n & n - 1 \end{array} \right | = (n + 1)(n - 1) - n^2 = n^2 - 1 - n^2  = -1$ \sspace
	П9) $\left | \begin{array}{rr} \cos{a} & -\sin{a}\\ \sin{a} & \cos{a} \end{array} \right | = \cos^2{a} + \sin^2{a} = 1$
	\sspace
	\textbf{4) П44, П47, П58} \sspace
	П44) $\left | \begin{array}{rrr} 3 & 2 & 1\\ 2 & 5 & 3\\ 3 & 4 & 2\end{array} \right | = 3 * 5 * 2 + 2 * 3 * 3 + 1 * 2 * 4 - 1 * 5 * 3 - 3 * 4 * 3 - 2 * 2 * 2 = 30 + 18 + 8 - 15 - 36 - 8 = -3$ \bs
	П47) $\left | \begin{array}{rrr} 3 & 4 & -5\\ 8 & 7 & -2 \\ 2 & -1 & 8\end{array} \right | = 3 * 7 * 8 + 4 * (-2) * 2 + (-5)*8*(-1) - (-5)*7*2 - (-2)*(-1)*3 - 8 * 4 * 8 = \sspace = 168 - 16 + 40 + 70 - 6 - 256 = 0$ \bs
	П58) $\left | \begin{array}{rrr} 0 & a & 0\\ b & c & d \\ 0 & e & 0\end{array} \right | = 0*c*d + a*d*0 + 0*b*e - 0*c*0 - d*e*0 - 0*a*b = 0$ \newpage
	\textbf{5) П190, П191} \sspace
	П190) $a_{27}*a_{36}*a_{51}*a_{74}*a_{25}*a_{43}*a_{62}$ \sspace
	Запишем соответствующую перестановку $\sigma$, проверим ее корректность \sspace
	$\sigma = \begin{pmatrix} 2 & 3 & 5 & 7  & 2 & 4 & 6  \\ 7 & 6 & 1 & 4  & 5 & 3 & 2 \end{pmatrix}$ \sspace
	Так как в $\sigma$ 2 переходит как в 7, так и в 5, то $\sigma$ не является перестановкой, значит исходное произведение не входит в определитель. \sspace
	П191) $a_{33}*a_{16}*a_{72}*a_{27}*a_{55}*a_{61}*a_{44}$ \sspace
	$\sigma = \begin{pmatrix} 3 & 1 & 7 & 2  & 5 & 6 & 4  \\ 3 & 6 & 2 & 7  & 5 & 1 & 4 \end{pmatrix}$ \sspace
	Так как $\sigma$ является переставкой, то исходное произведение входит в определитель. Посчитаем четность $\sigma$ через детерминант. \sspace
	$\sigma =  (16) * (27) * (3) * (4) * (5) \quad det(\sigma)  = 7 - 5 = 2 \quad sgn(\sigma) = 1$ \sspace
	\textbf{6) П197, П198} \sspace
	П197) $a_{62}*a_{i5}*a_{33}*a_{k4}*a_{46}*a_{21}$ \sspace
	$\sigma = \begin{pmatrix} 6 & i & 3 & k  & 4 & 2 \\ 2 & 5 & 3 & 4 & 6 & 1\end{pmatrix}$ \sspace
	Чтобы $\sigma$ была перестановкой, $i, k$ должны быть $1, 5$ или $5, 1$ соответственно. Так же можно заметить, что в 2 случаях знак перестановки будет противоположным, потому что отличается 1 транспозицией. Попробуем подставить 1 вариант, если знак будет положительным, значит подходит 2 вариант. \sspace
	$\sigma  = \begin{pmatrix} 6 &1 & 3 & 5  & 4 & 2 \\ 2 & 5 & 3 & 4 & 6 & 1\end{pmatrix} = (1,5,4,6,2)*(3) $ \sspace
	$det(\sigma) = 6 - 2 = 4 \quad sgn(\sigma) = 1$ \sspace
	1 вариант не подошел, значит подойдет второй, когда $i = 5, k = 1 $ \sspace
	П198) $a_{47}*a_{63}*a_{1i}*a_{55}*a_{7k}*a_{24}*a_{31}$ \sspace
	$\sigma = \begin{pmatrix} 4 & 6 & 1 & 5  & 7 & 2 & 3\\ 7 & 3 & i & 5 & k & 4 & 1\end{pmatrix}$ \sspace
	Возможные варианты: $(i, k) = (2, 6) \lor (i, k) = (6, 2)$ Попробуем первый \sspace
	$\sigma = \begin{pmatrix} 4 & 6 & 1 & 5  & 7 & 2 & 3\\ 7 & 3 & 2 & 5 & 6 & 4 & 1\end{pmatrix} = (1,2,4,7,6,3)*(5)$ \sspace
	$det(\sigma) = 7 - 2 = 5 \quad sgn(\sigma) = -1$ \sspace
	Значит подходит $(i,k) = (6, 2)$
	\sspace
	\textbf{7) Найдите коэффициент при $x^5$ в выражении определителя} \sspace
	$\left | \begin{array}{rrrrr} 2 & -3 & x & 4 & -5\\ 3 & x & -1 & -2 & 4\\ 1 & 3 & 1 & x & 1\\ -3 & x^2 & -1 & 1 & x\\ x & -2 & 4 & 5 & -2 \end{array} \right |$ \sspace
	Либо взять из каждой строчки по х, получим комбинацию, соответствующую перестановке $(1, 3,2,4,5) = (1, 3,4,5)*(2) \quad det = 5 - 2 = 3 \quad sgn = 1$. \\
	Либо можно взять из 4 строки $x^2$ и не взять в какой-то строчке x. Тут очевидно, в какой строчке не надо брать, это во второй, потому что тогда из второго столбца мы возьмем 2 элемента. Тогда остается только 1 вариант, взять из 2 строчки 4, из 4 строчки взять $x^2$, из остальных $x$. Получим комбинацию, соответствующую $(3, 5, 4, 2,1) = (1,3,4,2,5) \quad det = 5 - 1 = 4 \quad sgn = 1$ Коэффициент увеличится на 4. Итого коэффициент будет 5. \sspace
	\textbf{8) Найдите коэффициент при $x^4$ в выражении определителя} \sspace
	$\left | \begin{array}{rrrrr} 1 & 3 & x & 2 & 2\\ x & 2 & 1 & 4 & 5\\ x & 1 & x & 5 & x\\ 3 & x & 1& 2 & 3\\ 1 & 2 & 4 & x & 2 \end{array} \right |$ \sspace
	Важно заметить, что мы можем получить $\alpha x^4$, когда в 4 строках мы возьмем $x$, а в пятой возьмем не $x$. То есть мы однозначно определяем последний элемент, когда выбираем 4 $x$ в разных строчках. Также мы не можем не брать $x$ из 3 строчки, ведь тогда мы должны будем взять из 3 строчки 5 элемент, который $x$. Пусть мы не берем $x$ из первой строки, тогда мы можем взять из 3 строки $x$ на 3 или 5 позиции. В любом случае получим комбинации $2*x^4$. \textbf{Итого, суммарный коэффициент при $x^4$ 4.} Если мы не берем $x$ из 2 строчки, тогда остается только 1 вариант, который соответствует перестановке $(3,5,1,2,4)$. \textbf{Получаем коэффициент при $x^4$ = 5}. Пусть мы не $x$ из 4 строки, тогда вариантов вообще нет, потому что из 3 строки мы не можем взять первый $x$ из-за второго ряда, второй $x$ из-за первого ряда и третий из-за того, что тогда мы должны будем взять $x$ из 4 строки. Пусть мы не берем $x$ из 5 строки, тогда вариантов тоже нет. Получается, что итоговый суммарный коэффициент по всем способам равен $4 + 5 = 9$. \sspace
	9)Найдите наибольшее значение определителя матрицы 3 × 3, у которой все элементы равны 0 или 1. \sspace
	Для начала оценим максимальное значение определителя такой матрицы. Каждое произведение элементов в сумму определителя принимает значение либо 0, либо 1, но так как мы берем только половину из всех слагаемых со знаком +, максимальное значение определителя не может быть больше 3. Единственный случай, когда положительная часть суммы определителя равняется 3, когда вся матрица состоит из единиц. Но в таком случае определитель матрицы равен 0. То есть максимальное значение определителя не может быть 3, значит оно меньше либо равно 2. Приведем пример, когда определитель матрицы состоящей из нулей и единиц равняется 2. $A = \matthree{\rowthree{1}{1}{0}}{\rowthree{0}{1}{1}}{\rowthree{1}{0}{1}}$. \\
	Ответ: 2 
	
 \end{document}