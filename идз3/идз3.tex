\documentclass[12pt, a4paper]{article}
\usepackage{gset}
\setcounter{MaxMatrixCols}{50}
\begin{document}
	\maintitle{ЛАиГ ИДЗ 3}
	\textbf{1. Найдите матрицу, обратную к данной} \bs
	$
	\mtrfor{\rownine{2}{-3}{0}{-2}{|}{1}{0}{0}{0}}{\rownine{4}{7}{4}{1}{|}{0}{1}{0}{0}}{\rownine{-1}{-1}{-1}{0}{|}{0}{0}{1}{0}}{\rownine{0}{-3}{-1}{-1}{|}{0}{0}{0}{1}}
	\rsa \bs 
	\rsa \textbf{III} * (-1),  \textbf{I} \Leftrightarrow \textbf{III} \rsa
	\mtrfor{\rownine{1}{1}{1}{0}{|}{0}{0}{-1}{0}}{\rownine{4}{7}{4}{1}{|}{0}{1}{0}{0}}{\rownine{2}{-3}{0}{-2}{|}{1}{0}{0}{0}}{\rownine{0}{-3}{-1}{-1}{|}{0}{0}{0}{1}} 
	\rsa \bs 
	\rsa \textbf{II - 4I}, \textbf{III - 2I} \rsa \mtrfor{\rownine{1}{1}{1}{0}{|}{0}{0}{-1}{0}}{\rownine{0}{3}{0}{1}{|}{0}{1}{4}{0}}{\rownine{0}{-5}{-2}{-2}{|}{1}{0}{2}{0}}{\rownine{0}{-3}{-1}{-1}{|}{0}{0}{0}{1}}
	\rsa \bs
	\rsa \textbf{II + IV}, \textbf{III - 2IV} \rsa
	\mtrfor{\rownine{1}{1}{1}{0}{|}{0}{0}{-1}{0}}{\rownine{0}{0}{-1}{0}{|}{0}{1}{4}{1}}{\rownine{0}{1}{0}{0}{|}{1}{0}{2}{-2}}{\rownine{0}{-3}{-1}{-1}{|}{0}{0}{0}{1}}
	\rsa \bs
	\rsa \textbf{II} \Leftrightarrow \textbf{III}, \textbf{IV + 3II} \rsa
	\mtrfor{\rownine{1}{1}{1}{0}{|}{0}{0}{-1}{0}}{\rownine{0}{1}{0}{0}{|}{1}{0}{2}{-2}}{\rownine{0}{0}{-1}{0}{|}{0}{1}{4}{1}}{\rownine{0}{0}{-1}{-1}{|}{3}{0}{6}{-5}}
	\rsa \bs
	\rsa \textbf{IV - III} \rsa
	\mtrfor{\rownine{1}{1}{1}{0}{|}{0}{0}{-1}{0}}{\rownine{0}{1}{0}{0}{|}{1}{0}{2}{-2}}{\rownine{0}{0}{-1}{0}{|}{0}{1}{4}{1}}{\rownine{0}{0}{0}{-1}{|}{3}{-1}{2}{-6}}
	\rsa \bs
	\rsa \textbf{III} * (-1), \textbf{IV} * (-1) \rsa
	\mtrfor{\rownine{1}{1}{1}{0}{|}{0}{0}{-1}{0}}{\rownine{0}{1}{0}{0}{|}{1}{0}{2}{-2}}{\rownine{0}{0}{1}{0}{|}{0}{-1}{-4}{-1}}{\rownine{0}{0}{0}{1}{|}{-3}{1}{-2}{6}}
	\rsa \bs
	\rsa \textbf{I - II}, \textbf{I - III} \rsa
	\mtrfor{\rownine{1}{0}{0}{0}{|}{-1}{1}{1}{3}}{\rownine{0}{1}{0}{0}{|}{1}{0}{2}{-2}}{\rownine{0}{0}{1}{0}{|}{0}{-1}{-4}{-1}}{\rownine{0}{0}{0}{1}{|}{-3}{1}{-2}{6}}
	$
	\bs
	\answer{$\mtrfor{\rowfor{-1}{1}{1}{3}}{\rowfor{1}{0}{2}{-2}}{\rowfor{0}{-1}{-4}{-1}}{\rowfor{-3}{1}{-2}{6}}$}
	\newpage
	\textbf{2. Решить уравнение относительно незвестной перестановки X} \bs
	$
	\bigg(\mattwo{1 & 2 & 3 & 4 & 5 & 6 & 7 & 8}{6 & 5 & 7 & 8 & 3 & 1 & 4 & 2}^{19} * \mattwo{1 & 2 & 3 & 4 & 5 & 6 & 7 & 8}{3 & 6 & 8 & 7 & 4 & 1 & 2 & 5}^{-1}\bigg)^{141} * X = \mattwo{1 & 2 & 3 & 4 & 5 & 6 & 7 & 8}{4 & 1 & 3 & 6 & 7 & 2 & 8 & 5} \bs
	\mattwo{1 & 2 & 3 & 4 & 5 & 6 & 7 & 8}{6 & 5 & 7 & 8 & 3 & 1 & 4 & 2}^{19} = (1, 6)^{19}*(2, 5, 3, 7, 4, 8)^{19} = (1, 6)*(2, 5, 3, 7, 4, 8) = \bs = \mattwo{1 & 2 & 3 & 4 & 5 & 6 & 7 & 8}{6 & 5 & 7 & 8 & 3 & 1 & 4 & 2} \bs
	\mattwo{1 & 2 & 3 & 4 & 5 & 6 & 7 & 8}{3 & 6 & 8 & 7 & 4 & 1 & 2 & 5}^{-1} = \mattwo{1 & 2 & 3 & 4 & 5 & 6 & 7 & 8}{6 & 7 & 1 & 5 & 8 & 2 & 4 & 3} \bs
	\mattwo{1 & 2 & 3 & 4 & 5 & 6 & 7 & 8}{6 & 5 & 7 & 8 & 3 & 1 & 4 & 2} * \mattwo{1 & 2 & 3 & 4 & 5 & 6 & 7 & 8}{6 & 7 & 1 & 5 & 8 & 2 & 4 & 3} = \mattwo{1 & 2 & 3 & 4 & 5 & 6 & 7 & 8}{2 & 8 & 4 & 3 & 1 & 6 & 5 & 7} \bs
	\mattwo{1 & 2 & 3 & 4 & 5 & 6 & 7 & 8}{2 & 8 & 4 & 3 & 1 & 6 & 5 & 7}^{141} = (1, 2, 8, 7, 5)^{141}(3, 4)^{141}(6)^{141} = (1, 2, 8, 7, 5)(3, 4)(6) = \bs = 
	\mattwo{1 & 2 & 3 & 4 & 5 & 6 & 7 & 8}{2 & 8 & 4 & 3 & 1 & 6 & 5 & 7} \bs
	\answer{\mattwo{1 & 2 & 3 & 4 & 5 & 6 & 7 & 8}{2 & 8 & 4 & 3 & 1 & 6 & 5 & 7}}
	$
	\bs
	\textbf{3. Определить четность перестановки} \bs
	$
	\mattwo{1 & 2 & \cdots & 97 & 98 & \cdots & 115 & 116 & \cdots & 141}{45 & 46 & \cdots & 141 & 27 & \cdots & 44 & 1 & \cdots & 26} \sspace
	$
	Рассмотрим 1 блок от 45 до 141. Они будут образовывать инверсии со всеми числами из других блоков. То есть $97 * 44 = 4268$. Числа из второго блока (27-44) будут образовывать новые инверсии со всеми числами из последнего блока. То есть $18 * 26 = 468$. Последний блок не будет образовывать новых инверсий. Итого: $4268 + 468 = 4 736$. Так как инверсий четное число, то и вся перестановка четная. 
	\sspace
	\answer{Четная}
	\sspace
	PS. Вся перестановка это один цикл: \sspace \quad (1, 45, 89, 133, 18, 62, 106, 35, 79, 123, 8, 52, 96, 140, 25, 69, 113, 42, 86, 130, 15, 59, 103, 32, 76, 120, 5, 49, 93, 137, 22, 66, 110, 39, 83, 127, 12, 56, 100, 29, 73, 117, 2, 46, 90, 134, 19, 63, 107, 36, 80, 124, 9, 53, 97, 141, 26, 70, 114, 43, 87, 131, 16, 60, 104, 33, 77, 121, 6, 50, 94, 138, 23, 67, 111, 40, 84, 128, 13, 57, 101, 30, 74, 118, 3, 47, 91, 135, 20, 64, 108, 37, 81, 125, 10, 54, 98, 27, 71, 115, 44, 88, 132, 17, 61, 105, 34, 78, 122, 7, 51, 95, 139, 24, 68, 112, 41, 85, 129, 14, 58, 102, 31, 75, 119, 4, 48, 92, 136, 21, 65, 109, 38, 82, 126, 11, 55, 99, 28, 72, 116). \sspace Так что четность можно было посчитать и через детерминант, но такое решение мне показалось нечестным. 
	\bs
	\textbf{4. Вычислить определитель} \bs
	$
	\detsix{0 & 0 & 1 & 0 & 0 & x}{6 & 0 & 1 & 2 & 0 & 1}{0 & 6 & 0 & 2 & 3 & 6}{x & x & 0 & x & 6 & 6}{0 & x & 0 & 2 & 0 & x}{0 & 6 & 2 & 3 & 9 & 9}
	= 
	\detsix{0 & 0 & 1 & 0 & 0 & 0}{6 & 0 & 1 & 2 & 0 & 1 - x}{0 & 6 & 0 & 2 & 3 & 6}{x & x & 0 & x & 6 & 6}{0 & x & 0 & 2 & 0 & x}{0 & 6 & 2 & 3 & 9 & 9 - 2x}
	= \detfive{6 & 0 & 2 & 0 & 1 - x}{0 & 6 & 2 & 3 & 6}{x & x & x & 6 & 6}{0 & x & 2 & 0 & x}{0 & 6 & 3 & 9 & 9 - 2x}
	= \bs =
	\detfive{6 & 0 & 2 & 0 & 1 - x}{0 & 6 & 2 & 3 & 6}{0 & x & x - \frac{x}{3} & 6 & 6 - \frac{x(1 - x)}{6}}{0 & x & 2 & 0 & x}{0 & 6 & 3 & 9 & 9 - 2x}
	= 6 * \detfour{6 & 2 & 3 & 6}{x & x - \frac{x}{3} & 6 & 6 - \frac{x(1-x)}{6}}{x & 2 & 0 & x}{6 & 3 & 9 & 9 - 2x} 
	= \bs = 
	6 * \detfour{6 & 2 & 3 & 6}{x - 12 & x - \frac{x}{3} - 4& 0 & -6 - \frac{x(1-x)}{6}}{x & 2 & 0 & x}{6 & 3 & 9 & 9 - 2x}
	= 6 * \detfour{6 & 2 & 3 & 6}{x - 12 & x - \frac{x}{3} - 4& 0 & -6 - \frac{x(1-x)}{6}}{x & 2 & 0 & x}{-12& -3 & 0 & -9 - 2x}
	= \bs =
	18 * \detthree{x - 12 & \frac{2x}{3} - 4 & -6 - \frac{x(1-x)}{6}}{x & 2 & x}{-12 & -3 & -9-2x} = \bs = 18((x - 12)*2*(-9-2x) + (\frac{2x}{3} - 4)*x*(-12) + (-6 - \frac{x(1-x)}{6})*x*(-3)) - 18((-6 - \frac{x(1-x)}{6})*2*(-12) + x*(-3)*(x - 12) + (-9-2x)*(\frac{2x}{3}-4)*x) = 18(-4x^2 + 30x -24 - 8x^2 + 48x + 18x - \frac{x^2 - x^3}{2}) - 18(144 + 4x^2 - 4x - 3x^2 + 36x - 6x^2 + 36x - \frac{4x^3}{3} + 8x^2) = 18(\frac{x^3}{2} - \frac{25x^2}{2} + 96x - 24) - 18(-\frac{4x^3}{3} + 3x^2 + 68x + 144) = 18(\frac{11x^3}{6} - \frac{31x^2}{2} + 28x - 168) = 33x^3 - 279x^2 + 504x - 3024 \sspace
	\answer{33x^3 - 279x^2 + 504x - 3024} \bs
	$
	\textbf{5. Найдите коэффициент $x^5$ в выражении определителя} \bs
	Выпишем все перестановки где присутствует $x^5$ и коэффициенты, с которыми они входят в выражение определителя: \\
	Перестановка: 2431765, k: -10\\
	Перестановка: 2437165, k: 16\\
	Перестановка: 3412765, k: -9\\
	Перестановка: 3472165, k: 9\\
	Перестановка: 4132765, k: -10\\
	Перестановка: 4732165, k: 15\\
	Перестановка: 5132764, k: 16\\
	Перестановка: 5412763, k: 9\\
	Перестановка: 5431762, k: 15\\
	Перестановка: 5432176, k: -36\\
	Перестановка: 5432716, k: 36\\
	Перестановка: 5437162, k: -24\\
	Перестановка: 5472163, k: -9\\
	Перестановка: 5732164, k: -24\\
	Перестановка: 6432175, k: 36\\
	Перестановка: 6432715, k: -36\\
	Итого: -6 \\
	\answer{-6}
\end{document}