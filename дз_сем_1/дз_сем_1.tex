\documentclass[12pt, a4paper]{article}
\usepackage[utf8]{inputenc}
\usepackage[T2A]{fontenc}
\usepackage[english, russian]{babel}
\usepackage{amsmath, amsfonts, amssymb}
\usepackage{graphicx}
\newcommand{\B}[1]{\textbf{#1}}
\usepackage[left=2.50cm, right=2.00cm, top=2.00cm, bottom=2.00cm]{geometry}
\begin{document}
	\begin{center}
		\B{\underline{ЛАиГ. Домашнее задание 1}}
	\end{center}
	1) К17.1(а, б, г) \\
	\\
	а) 
	\begin{math}
		\left (
		\begin{array}{rr}
			1 & n\\
			0 & 1
		\end{array}
		\right )
	\end{math}
	$\cdot$ 
	\begin{math}
		\left (
		\begin{array}{rr}
			1 & m\\
			0 & 1
		\end{array}
		\right )
	\end{math}
	=
	\begin{math}
		\left (
		\begin{array}{rr}
			1 \cdot 1 + n \cdot 0 & 1 \cdot m + n \cdot 1\\
			0 \cdot 1 + 1 \cdot 0 & 0 \cdot m + 1 \cdot 1
		\end{array}
		\right )
	\end{math}
	=
	\begin{math}
		\left (
		\begin{array}{rr}
			1& m + n\\
			0 & 1
		\end{array}
		\right )
	\end{math}
	\\
	\\
	\\
	б)
	\begin{math}
		\left (
		\begin{array}{rr}
			\cos\alpha & -\sin\alpha\\
			\sin\alpha & \cos\alpha
		\end{array}
		\right )
	\end{math}
	$\cdot$
	\begin{math}
		\left (
		\begin{array}{rr}
			\cos\beta& -\sin\beta\\
			\sin\beta & \cos\beta
		\end{array}
		\right )
	\end{math}
	= 
	\begin{math}
		\left (
		\begin{array}{rr}
			\cos\alpha\cos\beta - \sin\alpha\sin\beta& -\cos\alpha\sin\beta - \sin\alpha\cos\beta\\
			\sin\alpha\cos\beta + \cos\alpha\sin\beta & -\sin\alpha\sin\beta + \cos\alpha\cos\beta
		\end{array}
		\right )
	\end{math}
	\\
	\\
	\\
	=
	\begin{math}
		\left(
		\begin{array}{rr}
			\cos(\alpha + \beta)& -\sin(\alpha+\beta)\\
			\sin(\alpha + \beta) & \cos(\alpha + \beta)
		\end{array}
		\right )
	\end{math}
	\\
	\\
	\\
	е)
	\begin{math}
		\left(
		\begin{array}{rrr}
			1& -1 & 3\\
			-1& 1 & -3\\
			2& -2 & 6
		\end{array}
		\right )
	\end{math}
	$\cdot$
	\begin{math}
		\left(
		\begin{array}{rrr}
			1& 5 & 2\\
			0& 3 & -1\\
			2& 1 & -1
		\end{array}
		\right )
	\end{math}
	= 
	\begin{math}
		\left(
		\begin{array}{rrr}
			7& 5 & 0\\
			-7& -5 & 0\\
			14& 10 & 0
		\end{array}
		\right )
	\end{math}
	\\
	\\
	\\
	2) К17.2(б)\\
	\\
	б)
	\begin{math}
		\left(
		\begin{array}{rrr}
			3& 0 & 2\\
			0& 1 & 3\\
			2& 2 & 0\\
			0 & 1 & 0
		\end{array}
		\right )
	\end{math}
	$\cdot$
	\begin{math}
		\left(
		\begin{array}{rrrr}
			1& 2 & -1 & 2\\
			-2& -1 & 1 & 2\\
			2& 1 & 1 & 2
		\end{array}
		\right )
	\end{math}
	+
	\begin{math}
		\left(
		\begin{array}{rrrr}
			0& -4 & 6  & 1\\
			2& 2 & -5 & -2\\
			2& -2 & 6 & 4\\
			1 & 3 & 0 & 1
		\end{array}
		\right )
	\end{math}
	\\\\\\
	=
	\begin{math}
		\left(
		\begin{array}{rrrr}
			7& 8 & -1 & 10\\
			4& 2 & 4 & 8\\
			-2& 2 & 0 & 8\\
			-2 & -1 & 1 & 2
		\end{array}
		\right )
	\end{math}
	+
	\begin{math}
		\left(
		\begin{array}{rrrr}
			0& -4 & 6  & 1\\
			2& 2 & -5 & -2\\
			2& -2 & 6 & 4\\
			1 & 3 & 0 & 1
		\end{array}
		\right )
	\end{math}
	=
	\begin{math}
		\left(
		\begin{array}{rrrr}
			7&  4& 5  & 11\\
			6& 4 & -1 & 6\\
			0& 0 & 6 & 12\\
			-1 & 2 & 1 & 3
		\end{array}
		\right )
	\end{math}
	\\\\\\
	3)
	\\\\
	A = 
	\begin{math}
		\left(
		\begin{array}{rr}
			2 & -1 \\
			1 & 0 \\
		\end{array}
		\right )
	\end{math}
	; B = 
	\begin{math}
		\left(
		\begin{array}{rr}
			a & b \\
			c & d \\
		\end{array}
		\right )
	\end{math}
	\\\\\\
	AB = 
	\begin{math}
		\left(
		\begin{array}{rr}
			2 \cdot a - c  & 2 \cdot b - d \\
			a & b \\
		\end{array}
		\right )
	\end{math}
	; BA = 
	\begin{math}
		\left(
		\begin{array}{rr}
			2 \cdot a + b  & -a \\
			2 \cdot c + d & -c \\
		\end{array}
		\right )
	\end{math}
	\\\\\\
	Для того, чтобы B была коммутирующей матрицей для А, необходимо поэлементное равенство матриц AB и BA.
	\\\\
	\begin{math}
		\begin{cases}
			2 \cdot a - c = 2 \cdot a + b\\
			2 \cdot b - d = -a\\
			a = 2 \cdot c + d \\
			b = -c \\
		\end{cases}
	\end{math}
	\begin{math}
		\begin{cases}
			b + c = 0\\
			a = d - 2 \cdot b\\
			a = 2 \cdot c + d \\
			b + c = 0 \\
		\end{cases}
	\end{math}
	\\\\\\\\\\\\\\
	Выразим все переменные через a, b. Тогда матрица B будет иметь вид:
	\\\\
	\begin{math}
		\left(
		\begin{array}{rr}
			a  & b \\
			-b & a + 2 \cdot b \\
		\end{array}
		\right )
	\end{math}
	\\\\\\
	Все матрицы такого вида будут коммутативными для матрицы А.
	\\\\\\
	4) П802
	\\\\
	Вычислим методом мат индукции. \\\\
	База индукции: \\\\
	\begin{math}
		\left(
		\begin{array}{rr}
			\cos\alpha  & -\sin\alpha \\
			\sin\alpha & \cos\alpha \\
		\end{array}
		\right ) ^1
	\end{math}
	=
	\begin{math}
		\left(
		\begin{array}{rr}
			\cos\alpha  & -\sin\alpha \\
			\sin\alpha & \cos\alpha \\
		\end{array}
		\right )
	\end{math}
	\\\\
	Предположение:
	\\\\
	\begin{math}
		\left(
		\begin{array}{rr}
			\cos\alpha  & -\sin\alpha \\
			\sin\alpha & \cos\alpha \\
		\end{array}
		\right ) ^n
	\end{math}
	=
	\begin{math}
		\left(
		\begin{array}{rr}
			\cos (n\alpha)  & -\sin (n\alpha) \\
			\sin (n\alpha) & \cos (n\alpha) \\
		\end{array}
		\right )
	\end{math}
	\\\\
	Шаг индукции :
	\\\\
	\begin{math}
		\left(
		\begin{array}{rr}
			\cos\alpha  & -\sin\alpha \\
			\sin\alpha & \cos\alpha \\
		\end{array}
		\right ) ^{n+1}
	\end{math}
	=
	\begin{math}
		\left(
		\begin{array}{rr}
			\cos\alpha  & -\sin\alpha \\
			\sin\alpha & \cos\alpha \\
		\end{array}
		\right ) ^{n}
	\end{math}
	$\cdot$
	\begin{math}
		\left(
		\begin{array}{rr}
			\cos\alpha  & -\sin\alpha \\
			\sin\alpha & \cos\alpha \\
		\end{array}
		\right )
	\end{math}
	=\\\\
	=
	\begin{math}
		\left(
		\begin{array}{rr}
			\cos (n\alpha)  & -\sin (n\alpha) \\
			\sin (n\alpha) & \cos (n\alpha) \\
		\end{array}
		\right )
	\end{math}
	$\cdot$
	\begin{math}
		\left(
		\begin{array}{rr}
			\cos\alpha  & -\sin\alpha \\
			\sin\alpha & \cos\alpha \\
		\end{array}
		\right )
	\end{math}
	=\\\\
	=
	\begin{math}
		\left(
		\begin{array}{rr}
			\cos (n\alpha) \cdot \cos\alpha - \sin (n\alpha) \cdot \sin\alpha & -2\sin (n\alpha)\cos\alpha \\
			2\sin (n\alpha)\cos\alpha & \cos(n\alpha)\cos\alpha - \sin (n\alpha)\sin\alpha \\
		\end{array}
		\right )
	\end{math}
	=\\\\
	=
	\begin{math}
		\left(
		\begin{array}{rr}
			\cos ((n + 1)\alpha)  & -\sin ((n + 1)\alpha) \\
			\sin ((n + 1)\alpha) & \cos ((n + 1)\alpha) \\
		\end{array}
		\right )
	\end{math}
	\\\\\\
	5) П803
	\\\\
	Вычислим методом мат индукции. 
	\\\\
	База индукции:
	\\\\
	\begin{math}
		\left(
		\begin{array}{rrrr}
			\lambda_1 &  &  & 0 \\
			 & \lambda_2 &  &  \\
			 & & \ddots &\\
			 0 & & & \lambda_n
		\end{array} 
		\right ) ^1
	\end{math}
	= 
	\begin{math}
		\left(
		\begin{array}{rrrr}
			\lambda_1 &  &  & 0 \\
			& \lambda_2 &  &  \\
			& & \ddots &\\
			0 & & & \lambda_n
		\end{array} 
		\right )
	\end{math}
	\newpage
	Предположение:
	\\\\
	\begin{math}
		\left(
		\begin{array}{rrrr}
			\lambda_1 &  &  & 0 \\
			& \lambda_2 &  &  \\
			& & \ddots &\\
			0 & & & \lambda_n
		\end{array} 
		\right ) ^k
	\end{math}
	=
	\begin{math}
		\left(
		\begin{array}{rrrr}
			\lambda_1^k &  &  & 0 \\
			& \lambda_2^k &  &  \\
			& & \ddots &\\
			0 & & & \lambda_n^k
		\end{array} 
		\right )
	\end{math}
	\\\\
	Шаг индукции:
	\\\\
	\begin{math}
		\left(
		\begin{array}{rrrr}
			\lambda_1 &  &  & 0 \\
			& \lambda_2 &  &  \\
			& & \ddots &\\
			0 & & & \lambda_n
		\end{array} 
		\right ) ^{k + 1}
	\end{math}
	=
	\begin{math}
		\left(
		\begin{array}{rrrr}
			\lambda_1 &  &  & 0 \\
			& \lambda_2 &  &  \\
			& & \ddots &\\
			0 & & & \lambda_n
		\end{array} 
		\right ) ^k
	\end{math}
	$\cdot$
	\begin{math}
		\left(
		\begin{array}{rrrr}
			\lambda_1 &  &  & 0 \\
			& \lambda_2 &  &  \\
			& & \ddots &\\
			0 & & & \lambda_n
		\end{array} 
		\right )
	\end{math}
	=\\\\\\
	=
	\begin{math}
		\left(
		\begin{array}{rrrr}
			\lambda_1^k &  &  & 0 \\
			& \lambda_2^k &  &  \\
			& & \ddots &\\
			0 & & & \lambda_n^k
		\end{array} 
		\right )
	\end{math}
	$\cdot$
	\begin{math}
		\left(
		\begin{array}{rrrr}
			\lambda_1 &  &  & 0 \\
			& \lambda_2 &  &  \\
			& & \ddots &\\
			0 & & & \lambda_n
		\end{array} 
		\right )
	\end{math}
	=
	\begin{math}
		\left(
		\begin{array}{rrrr}
			\lambda_1^{k+1} &  &  & 0 \\
			& \lambda_2^{k+1} &  &  \\
			& & \ddots &\\
			0 & & & \lambda_n^{k+1}
		\end{array} 
		\right )
	\end{math}
	\\\\\\
	6) П805\\\\\\
	Вычислим методом мат индукции
	\\
	\\
	\\
	База индукции:\\
	\\
	\begin{math}
		\left(
		\begin{array}{rr}
			\lambda  & 1 \\
			0 & \lambda\\
		\end{array}
		\right ) ^1
	\end{math}
	= 
	\begin{math}
		\left(
		\begin{array}{rr}
			\lambda  & 1 \\
			0 & \lambda\\
		\end{array}
		\right )
	\end{math}
	\\\\\\
	Предположение:
	\\\\\\
	\begin{math}
		\left(
		\begin{array}{rr}
			\lambda  & 1 \\
			0 & \lambda\\
		\end{array}
		\right ) ^n
	\end{math}
	=
	\begin{math}
		\left(
		\begin{array}{rr}
			\lambda^n  & n\cdot \lambda^{n - 1} \\
			0 & \lambda^n\\
		\end{array}
		\right ) 
	\end{math}
	\\\\\\
	Шаг индукции:
	\\\\\\
	\begin{math}
		\left(
		\begin{array}{rr}
			\lambda  & 1 \\
			0 & \lambda\\
		\end{array}
		\right ) ^{n + 1}
	\end{math}
	=
	\begin{math}
		\left(
		\begin{array}{rr}
			\lambda  & 1 \\
			0 & \lambda\\
		\end{array}
		\right ) ^n
	\end{math}
	$\cdot$
	\begin{math}
		\left(
		\begin{array}{rr}
			\lambda  & 1 \\
			0 & \lambda\\
		\end{array}
		\right )
	\end{math}
	=
	\begin{math}
		\left(
		\begin{array}{rr}
			\lambda^n  & n\cdot \lambda^{n - 1} \\
			0 & \lambda^n\\
		\end{array}
		\right ) 
	\end{math}
	$\cdot$
	\begin{math}
		\left(
		\begin{array}{rr}
			\lambda  & 1 \\
			0 & \lambda\\
		\end{array}
		\right )
	\end{math}
	=\\\\\\
	=
	\begin{math}
		\left(
		\begin{array}{rr}
			\lambda^{n + 1}  & \lambda^{n} \cdot 1 + n\cdot \lambda^{n - 1} \cdot \lambda \\
			0 & \lambda^{n + 1}\\
		\end{array}
		\right )
	\end{math}
	=
	\begin{math}
		\left(
		\begin{array}{rr}
			\lambda^{n + 1}  & (n + 1) \cdot \lambda^n \\
			0 & \lambda^{n + 1}\\
		\end{array}
		\right )
	\end{math}
	\newpage
	7) К17.5
	\\\\
	a) $f(x) = x^3 - 2x^2 + 1$
	; A = 
	\begin{math}
		\left(
		\begin{array}{rrr}
			2  & 1 & 0\\
			0 & 2 & 0\\
			1 & 1 & 1
		\end{array}
		\right )
	\end{math}
	\\\\\\
	\begin{math}
		A^2 = 
		\left(
		\begin{array}{rrr}
			4  & 4 & 0\\
			0 & 4 & 0\\
			3 & 4 & 1
		\end{array}
		\right )
	\end{math}
	; 
	\begin{math}
		A^3 = 
		\left(
		\begin{array}{rrr}
			8  & 12 & 0\\
			0 & 8 & 0\\
			7 & 12 & 1
		\end{array}
		\right )
	\end{math}
	\\\\\\
	$f(x) = 
		\left(
		\begin{array}{rrr}
			8  & 12 & 0\\
			0 & 8 & 0\\
			7 & 12 & 1
		\end{array}
		\right )
	 - 2 \cdot
	 	\left(
	 	\begin{array}{rrr}
	 		4  & 4 & 0\\
	 		0 & 4 & 0\\
	 		3 & 4 & 1
	 	\end{array}
	 	\right )
	  + E = 
	  \left (
	  \begin{array}{rrr}
	  	1  & 4 & 0\\
	  	0 & 1 & 0\\
	  	1&  4 & 0
	  \end{array}
	  \right )
	  $
	  \\\\\\
	  б) $f(x) = x^3 - 3x^2 + 2$
	  ; A = 
	  \begin{math}
	  	\left(
	  	\begin{array}{rrr}
	  		2  & 1 & 1\\
	  		1 &  2 & 1\\
	  		1 & 1 & 2
	  	\end{array}
	  	\right )
	  \end{math}
	  \\\\\\
	  \begin{math}
	  	A^2 = 
	  	\left(
	  	\begin{array}{rrr}
	  		6  & 5 & 5\\
	  		5 & 6 & 5\\
	  		5 & 5 & 6
	  	\end{array}
	  	\right )
	  \end{math}
	  ; 
	  \begin{math}
	  	A^3 = 
	  	\left(
	  	\begin{array}{rrr}
	  		22  & 21 & 21\\
	  		21 & 22 & 21\\
	  		21 & 21 & 22
	  	\end{array}
	  	\right )
	  \end{math}
	  \\\\\\
	  $f(x) = 
	  \left(
	  \begin{array}{rrr}
	  	22  & 21 & 21\\
	  	21 & 22 & 21\\
	  	21 & 21 & 22
	  \end{array}
	  \right )
	  - 3 \cdot
	  \left(
	  \begin{array}{rrr}
	  	6  & 5 & 5\\
	  	5 & 6 & 5\\
	  	5 & 5 & 6
	  \end{array}
	  \right )
	  + 2\cdot E = 
	  \left (
	  \begin{array}{rrr}
	  	6  & 6 & 6\\
	  	6 & 6 & 6\\
	  	6 & 6 & 6
	  \end{array}
	  \right )
	  $
	  \\\\\\
	  8) К17.7
	  \\\\\\
	  Вычислим методом мат индукции
	  \\\\\\
	  База индукции: 
	  \\\\\\ 
	  \begin{math}
	  	H  ^ 1 _ {n,n} = 
	  	\left(
	  	\begin{array}{ccccc}
	  		0 & 1 & 0 & \cdots &0 \\
	  		0& 0 & 1 & \cdots & 0 \\
	  		\vdots& \vdots & \ddots & \vdots & \vdots \\
	  		0 & 0 & 0 & \ddots & 1\\
	  		0 & 0& 0& \cdots & 0
	  	\end{array} 
	  	\right )
	  \end{math}
	  \newpage
	  Предположение:
	  \\\\\\
	  \begin{math}
	  	H  ^ k = 
	  	\left(
	  	\begin{array}{rr}
	  		0 & E_{n - k, n - k}\\
	  		0 & 0
	  	\end{array} 
	  	\right )
	  \end{math}
	  \\\\\\
	  Шаг индукции:
	  \\\\\\
	  \begin{math}
	  	H  ^ {k + 1} = 
	  	\left(
	  	\begin{array}{lll}
	  		0 & \cdots & E_{n - k, n - k}\\
	  		0 & \ddots &  \vdots \\
	  		0 & \cdots & 0
	  	\end{array} 
	  	\right )
	  	\cdot 
	  	\left(
	  	\begin{array}{rr}
	  		0 & E_{n - 1, n - 1}\\
	  		0 & 0
	  	\end{array} 
	  	\right )
	  	=
	  	\left(
	  	\begin{array}{lll}
	  		0 & \cdots & E_{n - k - 1, n - k - 1}\\
	  		0 & \ddots &  \vdots \\
	  		0 & \cdots & 0
	  	\end{array} 
	  	\right )
	  \end{math}
	  \\\\\\
	  Переход возможен, пока k < n. Иначе на месте E стоит 0
	  \\\\\\
	  9) П829
	  \\\\\\
	  $A^2 - (a + d)A + ad - bc = 0$
	  ; 
	   \begin{math}
	   	A = 
	   	\left(
	   	\begin{array}{rr}
	   		a  & b \\
	   		c & d \\
	   	\end{array}
	   	\right )
	   \end{math}
	   \\\\\\
	   \begin{math}
	   	A^2 = 
	   	\left(
	   	\begin{array}{rr}
	   		a^2 + bc  & ab + bd \\
	   		ac  + bd & bc + d ^ 2 \\
	   	\end{array}
	   	\right )
	   \end{math}
	   \\\\\\
	   \begin{math}
	    (a + d)A = 
	   	\left(
	   	\begin{array}{rr}
	   		a ^ 2  + ad & ab + bd \\
	   		ac + cd & ad + d ^ 2 \\
	   	\end{array}
	   	\right )
	   \end{math}
	   \\\\\\
	   \begin{math}
	   	A ^ 2 - (a + d)A = 
	   	\left(
	   	\begin{array}{rr}
	   		bc -  ad & 0 \\
	   		0 & bc - ad \\
	   	\end{array}
	   	\right )
	   \end{math}
	   \\\\\\
	   \begin{math}
	   	\left(
	   	\begin{array}{rr}
	   		bc -  ad & 0 \\
	   		0 & bc - ad \\
	   	\end{array}
	   	\right )
	   	+
	   	\left(
	   	\begin{array}{rr}
	   	    ad & 0 \\
	   		0 & ad \\
	   	\end{array}
	   	\right )
	   	+
	   	\left(
	   	\begin{array}{rr}
	   		-bc & 0 \\
	   		0 & -bc \\
	   	\end{array}
	   	\right )
	   	= 0
	   \end{math}
	   \\\\\\
	   10) 
	   \\\\\\
	   \begin{math}
	   	A_{n,n} = 
	   	\left(
	   	\begin{array}{rrrr}
	   		a_{1, 1} & a_{1, 2} & \cdots & a_{1, n}\\
	   		a_{2, 1} & a_{2, 2} & \cdots & a_{2, n}\\
	   		\vdots & \vdots & \ddots & \vdots\\
	   		a_{n, 1} & a_{n, 2} & \cdots & a_{n, n}
	   	\end{array}
	   	\right )
	   \end{math}
	   ; 
	    \begin{math}
	   	E_{i,j} = 
	   	\left(
	   	\begin{array}{rrrrr}
	   		0 & \cdots & \cdots & \cdots &  0\\
	   		\vdots & \ddots & \cdots & \cdots & \vdots\\
	   		\vdots & \cdots & e_{i, j}  = 1& \cdots & \vdots\\
	   		\vdots & \cdots & \cdots & \ddots & \vdots\\
	   		0 & \cdots & \cdots& \cdots & 0
	   		
	   	\end{array}
	   	\right )
	   \end{math}
	   \\\\\\
	    \begin{math}
	   	E_{i,j} \cdot A =  
	   	\left(
	   \begin{array}{ccccc}
		   	0 & \cdots & \cdots & \cdots & 0\\
		   	\vdots & \cdots & \cdots & \cdots & \vdots \\
		   	(i, 1) = a_{j, 1} & (i, 2) = a_{j, 2} & \cdots & (i, j - 1) = a_{j, n - 1} & (i, n) = a_{j, n} \\
		   	\vdots & \cdots & \cdots & \cdots & \vdots\\
		   	0 & \cdots & \cdots & \cdots & 0
	   \end{array}
	   	\right )
	   \end{math}
	   \\\\\\
	   \begin{math}
	   	A \cdot E_{i, j} = 
	   	\left(
	   	\begin{array}{ccccc}
	   		0 & \cdots & (1, j) = a_{1, i} & \cdots & 0\\
	   		\vdots & \cdots & (2, j) = a_{2, i} & \cdots & \vdots \\
	   		\vdots & \vdots & \vdots & \vdots & \vdots \\
	   		\vdots & \cdots & (n - 1, j) = a_{n - 1, i} & \cdots & \vdots \\
	   		0 & \cdots & (n, j)  = a_{n, i} & \cdots & 0
	   	\end{array}
	   	\right )
	   \end{math}
	   
	   \end{document}