\documentclass[12pt, a4paper]{article}
\usepackage{gset}
\setcounter{MaxMatrixCols}{50}
\begin{document}
	\maintitle{ЛАиГ. Домашнее задание 9-10}
	1) Пусть $V$ - векторное пространство. Докажите, что для всякого вектора $x \in V$ справедливы равенства а) $0 * x = \vec{0}$, б) $(-1) * x = -x$. \\

	а) $0 * x = (x - x) * x = x * x - x * x = \vec{0}$ \\
	
	б) $(-1) * x = -x$ \\
	
	Прибавим к каждой части уравнения $x$ \\
	
	$(-1) * x + x = \vec{0}$ \\
	
	$x * (-1 + 1) = \vec{0}$ \\
	
	$x * 0 = \vec{0}$ - верно по пункту (а) \sspace
	2) К35.1 Выяснить, является ли подпространством совокупность: \sspace
	В данной задаче $F$ - поле, $V$ - пространство, $U$ - подпространство. \\
	
	г) $F = R, V = R^2, U = \{(x, y): x \geq 0, y \geq 0\}$
	
	
\end{document}